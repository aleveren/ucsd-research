\documentclass{article}

\usepackage[final]{nips_2016_modified}

\usepackage[utf8]{inputenc} % allow utf-8 input
\usepackage[T1]{fontenc}    % use 8-bit T1 fonts
\usepackage{hyperref}       % hyperlinks
\usepackage{url}            % simple URL typesetting
\usepackage{booktabs}       % professional-quality tables
\usepackage{amsmath}
\usepackage{amsfonts}       % blackboard math symbols
\usepackage{nicefrac}       % compact symbols for 1/2, etc.
\usepackage{microtype}      % microtypography
%\usepackage{algorithm}
%\usepackage{algpseudocode}
\usepackage{graphicx}
\usepackage{float}
\usepackage{caption}
\usepackage{geometry}
\geometry{margin=1.5in}

\title{Hierarchical Topic Models}

\author{
  Andrew Leverentz \\
  \texttt{aleveren@eng.ucsd.edu} \\
}

\begin{document}

\maketitle

\begin{abstract}
In this article, we survey techniques for automatically inferring hierarchical structures from unlabeled collections of text documents.
We begin with a brief overview of probabilistic topic modeling, as implemented in models such as Probabilistic Latent Semantic Indexing (pLSI) and Latent Dirichlet Allocation (LDA).
After discussing some limitations of these models, we discuss two lines of research which have extended the LDA model and attempted to address some of its weaknesses.
One approach, based on Dirichlet Processes, treats topics as belonging to a potentially infinite tree; in this context, Bayesian non-parametric techniques are used to automatically select the structure of the tree.
The second approach, called Hierarchical Pachinko Allocation Modeling, uses a directed acyclic graph to recursively define topics as distributions not only over words but also over other topics.
We compare and contrast these approaches and discuss several inference algorithms that have been published in the literature.
We also discuss methods for visualizing, interpreting, and evaluating hierarchies of topics.
\end{abstract}

\section{Introduction}
TODO
In \cite{paisley2015nested}, there is some stuff.
Similarly, in \cite{blei2010nested}, there is some stuff.

TODO

\section{Conclusion}
TODO

%%%%%%%%%%%%%

\nocite{*}
%\bibliographystyle{plainnat}
\bibliographystyle{plain}
\bibliography{../bibliography}

\end{document}
