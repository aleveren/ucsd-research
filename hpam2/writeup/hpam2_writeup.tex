\documentclass{article}

\usepackage{fullpage}
\usepackage[utf8]{inputenc} % allow utf-8 input
\usepackage[T1]{fontenc}    % use 8-bit T1 fonts
\usepackage{hyperref}       % hyperlinks
\usepackage{url}            % simple URL typesetting
\usepackage{booktabs}       % professional-quality tables
\usepackage{amsmath}
\usepackage{amssymb}
\usepackage{amsfonts}       % blackboard math symbols
\usepackage{mathtools}
\usepackage{nicefrac}       % compact symbols for 1/2, etc.
\usepackage{microtype}      % microtypography
\usepackage{algorithm}
\usepackage{algpseudocode}
\usepackage{graphicx}
\usepackage{float}
\usepackage{caption}
\usepackage{outlines}
\usepackage{tikz}
\usepackage{dsfont}
\usepackage{empheq}
\usepackage{xfrac}
\usepackage{enumitem}
\usepackage{amsthm}
%\usepackage{geometry}
%\geometry{margin=1.5in}

\theoremstyle{definition}

\newtheorem{thm}{Theorem}
\newtheorem{lemma}[thm]{Lemma}
\newtheorem{claim}[thm]{Claim}
\newtheorem{defn}[thm]{Definition}

\newcommand{\nth}{^{\text{th}}}
\newcommand{\len}{\text{len}}
\newcommand{\indicator}{\mathds{1}}
\newcommand{\hackystatei}[1]{\State \parbox[t]{\dimexpr\linewidth-\algorithmicindent}{#1\strut}}
\newcommand{\hackystateii}[1]{\State \parbox[t]{\dimexpr\linewidth-\algorithmicindent-\algorithmicindent}{#1\strut}}
\newcommand{\argmax}{\mathop{\mathrm{argmax}}}
\newcommand{\argmin}{\mathop{\mathrm{argmin}}}

\newcommand{\Dirichlet}{\text{Dirichlet}}
\newcommand{\Categorical}{\text{Categorical}}
\newcommand{\Exit}{\textsc{Exit}}

\newcommand{\TODOcite}{[?]}

%\allowdisplaybreaks

\title{Inferring Hierarchies of Topics from Co-Occurrence Probabilities}

\author{
  Andrew Leverentz \\
  \texttt{aleveren@eng.ucsd.edu} \\
}

\date{}

\begin{document}

\maketitle

%%%%%%%%%%%%%%%%%%%%%%%%%%%

\section{Introduction}

Traditional probabilistic topic models, such as Latent Dirichlet Allocation (LDA) \TODOcite{}, treat observed documents as mixtures of latent topics, which are in turn viewed as probability distributions over a vocabulary of words.
In LDA, the topics themselves are assumed to co-occur with each other within documents by drawing topic proportions from a symmetric Dirichlet.
However, one may wish to view topics as being members of a hierarchy, in which topics can have varying levels of ``specificity'' or ``generality,'' and in which sub-topics are more likely to occur with super-topics (corresponding to ancestors in the hierarchy) than they are to occur with non-ancestors.

Several models exist which generalize the ``flat'' collections of topics produced by LDA into hierarchies.
In one class of models, we specify \emph{a priori} a fixed tree (or, more generally, a fixed DAG), and we infer a topic to associate with each leaf node.
In a second class of models, neither the topics nor the tree structure are fixed prior to inference, and Bayesian non-parametric techniques are used to discover a tree of potentially unbounded size.
Here, we will consider Pachinko Allocation Modeling (PAM) \TODOcite{} as a representative of the first type, and the Nested Hierarchical Dirichlet Process (NHDP) \TODOcite{} as a representative of the second type.
NHDP has the advantage of not requiring knowledge of the true structure of the topic hierarchy prior to inference;
however, because its underlying probabilistic model is significantly more complex than PAM, inference for NHDP is relatively cumbersome.

This paper defines a new method for deriving a hierarchy of topics which attempts to address some of the shortcomings of both of the above classes of models.
Unlike PAM, our model does not require specifying a fixed tree structure \emph{a priori}, but our method is significantly simpler and faster than the inference algorithms currently available for NHDP.
Moreover, we offer proof that, under certain assumptions on the data-generating process, our method is capable of exactly recovering the ``true'' tree of topics (with high probability given sufficient data).

\section{A Motivating Observation}

The ``anchor words'' algorithm \TODOcite{} is a non-probabilistic alternative to traditional LDA inference algorithms such as Gibbs sampling or variational inference.
Its main appeals are (1) its relatively fast runtime, (2) provable guarantees on its ability to recover the topics with which a given dataset was generated, and (3) its ability to handle non-trivial co-occurrence patterns.
The anchor-words algorithm produces a matrix $A$ representing the topics appearing in the corpus (each row of $A$ corresponds to a probability distribution over the vocabulary).
In addition, it produces a matrix $R$ representing the probabilities of pairs of topics co-occurring within a single document.
One of the observations that motivated this paper was that, when a corpus is generated according to a hierarchical topic model such as PAM, the $R$ matrix recovered by the anchor-words algorithm appears to contain significant information about the structure of the underlying hierarchy.
(TODO: include a figure demonstrating the structure that is visible in the $R$ matrix.)
Thus, a natural question to ask is, can we recover the structure of a hierarchy of topics based purely on observations of pairwise topic co-occurrence probabilities?

It is important to note that the motivating example obtains $R$ via the anchor-words algorithm, but there are other methods of estimating the $R$ matrix of topic co-occurrences.
For example, we could run a Gibbs sampler or variational inference for LDA, and then measure the empirical rates of topic co-occurrence.

\section{A Proposed Algorithm}

We propose the following algorithm for recovering a hierarchy of topics from a corpus of data.

\newcommand{\Rbar}{\overline{R}}

\begin{itemize}
  \item Input: A corpus of documents, $X$, and a desired number of topics, $K$.
  \item Estimate $A$, a matrix of topic definitions (for example, this can be done via anchor-words or LDA)
  \item Estimate $R$, the matrix of pairwise topic co-occurrences (again, via anchor-words or LDA)
  \item Compute a ratio matrix according to
  \begin{align}
  \Rbar_{ij} = \frac{R_{ij}}{ \bigl( \sum_{i'} R_{i'j} \bigr) \bigl( \sum_{j'} R_{ij'} \bigr) }
  \end{align}
  \item Specify a list of triplet constraints such that $(\{i, j\}, k)$ is included iff $\Rbar_{ij} > \Rbar_{jk}$ and $\Rbar_{ij} > \Rbar_{ik}$ (with $i$, $j$, and $k$ representing 3 distinct indices from $1$ to $K$).
  \item For each constraint, define its ``strength'' to be $\Rbar_{ij} - \max(\Rbar_{jk}, \Rbar_{ik})$.
        Sort the list of constraints by descending strength.
  \item Apply Aho's algorithm to the list of constraints to generate a tree in which the leaves correspond to the integers $1$ to $K$.
  \item If the previous step fails (i.e., returns a null tree), perform binary search to find the longest prefix of the constraint list which yields a non-null tree when applying Aho's algorithm.
  \item Return the resulting tree, along with the matrix of topics ($A$).
\end{itemize}

TODO: also propose a weighted version of Aho's algorithm (weighted by constraint-strength, as opposed to the above ``thresholded'' version)

\section{Proving Properties of the Algorithm}

TODO (adapt from HPAM2 writeup)

\section{Experimental Results on Simulated Data}

TODO

%%%%%%%%%%%%%%%%%%%%%%%%%%%%%%%%

\clearpage

{\bf TODO: generalize the following.} \\
Recall that the HPAM2 model is a special case of the generalized PAM model.
(HPAM2 adds the requirement that every internal node either has only leaf-node children or has exactly one leaf-node child.)

\section{Defining the HPAM2 Model}

In the hierarchical topic model HPAM2 (Hierarchical Pachinko Allocation Modeling, variant 2), we are given a rooted, directed acyclic graph $G$ and a collection of Dirichlet parameters.
Specifically, we have a vector $\alpha_\nu$ for each non-terminal node $\nu$, where the dimension of $\alpha_\nu$ is $1 + |\text{children}(\nu)|$.
Note that $\alpha_\nu$ is not defined for terminal nodes (i.e., nodes with no outgoing edges).
We consider the vector $\alpha_\nu$ to be indexed by the set $\{0\} \cup \text{children}(\nu)$.
We also have a vector $\eta$ which is the parameter for a $V$-dimensional Dirichlet, where $V$ is the size of our vocabulary.

The generative model for sampling a corpus of documents is as follows:
For each node $\nu \in G$ (including terminal nodes), sample $\beta_\nu$ according to $\text{Dirichlet}(\eta)$.
For each document $d$ and for each non-terminal node $\nu$, sample $\theta_{d,\nu}$ according to $\text{Dirichlet}(\alpha_\nu)$.
Next, for each word-slot $n$ in document $d$ and for each non-terminal node $\nu$, sample $z_{d,\nu,n}$ according to $\text{Categorical}(\theta_{d,\nu})$; here, we are choosing whether to exit the DAG early (by selecting $0$), or to continue further down the DAG by selecting a child of $\nu$.
Next, we can define, for each document $d$ and each word-slot $n$, a path $\pi_{d,n}$ through $G$ beginning at the root node.
Specifically, the first element of $\pi_{d,n}$ is always the root node of $G$, and we repeatedly extend $\pi_{d,n}$ according to the sampled value of $z_{d,\nu,n}$, where $\nu$ represents the previously-selected node.
If we ever encounter a selection $z_{d,\nu,n} = 0$, we terminate the path early.
Finally, we sample a word from the vocabulary $t_{d,n}$ according to $\text{Dirichlet}(\beta_{\Exit(\pi_{d,n})})$.
Here, $\Exit(\pi_{d,n})$ is the node in $G$ at which the path $\pi_{d,n}$ terminates, which may or may not be a terminal node.

\section{Analysis of Topic Co-occurrence Probabilities in HPAM2}

Next, we compute topic co-occurrence probabilities for the HPAM2 model.
Specifically, we wish to know the probability that a single document contains some pair of topics $\nu_1$ and $\nu_2$:
\begin{align}
p(\Exit(\pi_{d,1}) = \nu_1,
  \Exit(\pi_{d,2}) = \nu_2)
&=
\sum_{
  \substack{
    \tilde\pi_1 : \Exit(\tilde\pi_1) = \nu_1
    \\
    \tilde\pi_2 : \Exit(\tilde\pi_2) = \nu_2
  }
}
p(\pi_{d,1} = \tilde\pi_1, \pi_{d,2} = \tilde\pi_2)
\end{align}
Next, let $\nu \to \nu' \in \tilde\pi_1$ denote the set of transitions $(\nu, \nu')$ appearing in $\tilde\pi_1$.
Furthermore, let $\nu \to * \in \tilde\pi_1 \cup \tilde\pi_2$ denote the set of nodes $\nu$ for which there exists some transition $\nu \to \nu'$ appearing in either $\tilde\pi_1$ or $\tilde\pi_2$.
Let $\int_{\tilde\theta \in \Theta(\tilde\pi_1, \tilde\pi_2)}$ denote an iterated integral over $\tilde\theta_\nu$ for each $\nu \to * \in \tilde\pi_1 \cup \tilde\pi_2$.
With this notation, we can write the co-occurrence probability of two paths as follows:
\begin{align}
p(\pi_{d,1} = \tilde\pi_1, \pi_{d,2} = \tilde\pi_2)
&=
\int_{\tilde\theta \in \Theta(\tilde\pi_1, \tilde\pi_2)}
  \left(
    \prod_{\nu \to * \in \tilde\pi_1 \cup \tilde\pi_2}
    p(\theta_{d,\nu} = \tilde\theta_\nu)
  \right)
  \\ &\phantom{=}\qquad \cdot
  \left(
    \prod_{\nu \to \nu' \in \tilde\pi_1}
    p(z_{d,\nu,1} = \nu' | \theta_{d,\nu} = \tilde\theta_\nu)
  \right)
  \\ &\phantom{=}\qquad \cdot
  \left(
    \prod_{\nu \to \nu' \in \tilde\pi_2}
    p(z_{d,\nu,2} = \nu' | \theta_{d,\nu} = \tilde\theta_\nu)
  \right)
\\
&=
\int_{\tilde\theta \in \Theta(\tilde\pi_1, \tilde\pi_2)}
  \left(
    \prod_{\nu \to * \in \tilde\pi_1 \cup \tilde\pi_2}
    \Dirichlet(\tilde\theta_\nu | \alpha_\nu)
  \right)
  %\\ &\phantom{=}\qquad \cdot
  \left(
    \prod_{\nu \to \nu' \in \tilde\pi_1}
    \tilde\theta_{\nu,\nu'}
  \right)
  %\\ &\phantom{=}\qquad \cdot
  \left(
    \prod_{\nu \to \nu' \in \tilde\pi_2}
    \tilde\theta_{\nu,\nu'}
  \right)
\end{align}
Then, if we define
\begin{align}
S(\nu) &= \{0\} \cup \text{children}(\nu), \\
N(\nu,\nu',\{\tilde\pi_1,\tilde\pi_2\}) &= \text{number of times $\nu \to \nu'$ appears in either $\tilde\pi_1$ or $\tilde\pi_2$}, \\
M(\nu,\{\tilde\pi_1,\tilde\pi_2\}) &= \text{number of times $\nu \to *$ appears in either $\tilde\pi_1$ or $\tilde\pi_2$},
\end{align}
we have
\begin{align}
p(\pi_{d,1} = \tilde\pi_1, \pi_{d,2} = \tilde\pi_2)
&=
\prod_{\nu \to * \in \tilde\pi_1 \cup \tilde\pi_2}
\left[
\frac
  {
    \prod_{\nu' \in S(\nu)}
    \prod_{k=0}^{N(\nu,\nu',\{\tilde\pi_1,\tilde\pi_2\}) - 1}
    (k + \alpha_{\nu,\nu'})
  }
  {
    \prod_{k=0}^{M(\nu,\{\tilde\pi_1,\tilde\pi_2\}) - 1}
    \left(
      k + \sum_{\nu' \in S(\nu)}\alpha_{\nu,\nu'}
    \right)
  }
\right]
\label{eqn:pairPathProb}
\end{align}

Note that this formula easily generalizes beyond pairwise co-occurrences.
For example, we can compute $n$-way path co-occurrence probabilities as follows:
\begin{align}
p(\pi_{d,1} = \tilde\pi_1, \ldots, \pi_{d,n} = \tilde\pi_n)
&=
\prod_{\nu \to * \in \tilde\pi_1 \cup \ldots \cup \tilde\pi_n}
\left[
\frac
  {
    \prod_{\nu' \in S(\nu)}
    \prod_{k=0}^{N(\nu,\nu',\{\tilde\pi_1,\ldots,\tilde\pi_n\}) - 1}
    (k + \alpha_{\nu,\nu'})
  }
  {
    \prod_{k=0}^{M(\nu,\{\tilde\pi_1,\ldots,\tilde\pi_n\}) - 1}
    \left(
      k + \sum_{\nu' \in S(\nu)}\alpha_{\nu,\nu'}
    \right)
  }
\right]
\label{eqn:manyPathProb}
\end{align}
Furthermore, we can view the formula for single-path probabilities as a special case of the $n$-way co-occurrence formula:
\begin{align}
p(\pi_{d,1} = \tilde\pi)
&=
\prod_{\nu \to \nu' \in \tilde\pi}
\left[
\frac
  {
    \alpha_{\nu,\nu'}
  }
  {
    \sum_{\nu'' \in S(\nu)}\alpha_{\nu,\nu''}
  }
\right]
%=
%\prod_{\nu \to \nu' \in \tilde\pi}
%E[\theta_{d,\nu}]_{\nu'}
\label{eqn:singlePathProb}
\end{align}

\section{Properties of the Co-occurrence Matrix}

For an HPAM2 model with a corresponding tree $\mathcal T$, we define a matrix $C$ as follows:
\begin{align}
C_{i,j} &= \frac{p(\Exit(\pi_{d,1}) = i, \Exit(\pi_{d,2}) = j)}
                {p(\Exit(\pi_{d,1}) = i) \, p(\Exit(\pi_{d,2}) = j)}
\end{align}
Here, $i$ and $j$ are indices over the nodes of $\mathcal T$.

When the DAG underlying the HPAM2 model is a tree, the pairwise probabilities in \eqref{eqn:pairPathProb} can be simplified.
In particular, if $i$ and $j$ are arbitrary nodes in $\mathcal T$, we can rewrite the product over edges in \eqref{eqn:pairPathProb} by decomposing it into the edges that lead to the least common ancestor of $i$ and $j$, the edges that occur at the split, and the edges along each of the totally distinct paths following the split:
\begin{align}
p(\Exit(\pi_{d,1}) = i, \Exit(\pi_{d,2}) = j)
={}&
\phantom{{}\cdot{}} p(\text{edges before split}) \\
& \cdot p(\text{edges at split}) \nonumber \\
& \cdot p(\text{edges after split on path to $i$}) \nonumber \\
& \cdot p(\text{edges after split on path to $j$}) \nonumber
\end{align}

In the case where $i = j$, the paths never diverge (that is, there is no split), and so only the first factor contributes to the final product.
Thus, for the remainder of this analysis, we assume $i \neq j$.

Let the path from the root to $i$ be denoted
\begin{align}
\nu_0 \to \nu_1 \to \cdots \to \nu_s \to \nu^{(1)}_1 \to \cdots \to \nu^{(1)}_{L_1},
\end{align}
where $\nu^{(1)}_{L_1} = i$ and $\nu_s$ is the least common ancestor of $i$ and $j$.
Similarly, let the path from the root to $j$ be denoted
\begin{align}
\nu_0 \to \nu_1 \to \cdots \to \nu_s \to \nu^{(2)}_1 \to \cdots \to \nu^{(2)}_{L_2},
\end{align}
where $\nu^{(2)}_{L_2} = j$.
Note that $\nu_0 \to \nu_1 \to \cdots \to \nu_s$ denotes the ``shared'' portion of the paths to $i$ and $j$.

With this notation, we have
\begin{align}
p(\text{edges before split})
&=
\prod_{k=1}^s \frac
  {\left( 1 + \alpha_{\nu_{k-1} \nu_k} \right) \alpha_{\nu_{k-1} \nu_k} }
  {\left( 1 + \sum \alpha_{\nu_{k-1}} \right) \sum \alpha_{\nu_{k-1}} }
\\
p(\text{edges at split})
&= \frac
  {\alpha_{\nu_s \nu^{(1)}_1} \, \alpha_{\nu_s \nu^{(2)}_1}}
  { \left( 1 + \sum \alpha_{\nu_s} \right) \sum \alpha_{\nu_s} }
\\
p(\text{edges after split on path to $i$})
&=
\prod_{k=2}^{L_1} \frac
  {\alpha_{\nu^{(1)}_{k-1} \nu^{(1)}_k} }
  { \sum \alpha_{\nu^{(1)}_{k-1}} }
\label{eqn:singleFactorI}
\\
p(\text{edges after split on path to $j$})
&=
\prod_{k=2}^{L_2} \frac
  {\alpha_{\nu^{(2)}_{k-1} \nu^{(2)}_k} }
  { \sum \alpha_{\nu^{(2)}_{k-1}} }
\label{eqn:singleFactorJ}
\end{align}

We can now derive a closed-form expression for the entries of $C$.
When we combine the above results, many factors cancel, yielding
\begin{align}
C_{i,j}
&=
\begin{cases}
\left(
  \prod_{k=1}^s
  \frac { 1 + \alpha_{\nu_{k-1} \nu_k} } { \alpha_{\nu_{k-1} \nu_k} }
\right)
\cdot
\left(
  \prod_{k=0}^s
  \frac { \sum \alpha_{\nu_k} } { 1 + \sum \alpha_{\nu_k} }
\right),
& i \neq j, \\
\prod_{k=1}^s \frac
  { \left( 1 + \alpha_{\nu_{k-1} \nu_k} \right) \sum \alpha_{\nu_{k-1}} }
  { \left( 1 + \sum \alpha_{\nu_{k-1}} \right) \alpha_{\nu_{k-1} \nu_k} },
& i = j.
\end{cases}
\label{eqn:closedFormC}
\end{align}

\begin{defn}[Limited-Sparsity Property]
Suppose an HPAM2 model based on a tree $\mathcal T$ has a Dirichlet parameter vector $\alpha_\nu$ specified for each internal node $\nu$.
We say this model satisfies the \emph{limited-sparsity property} if, for each edge between internal nodes $i \to j$, we have $\sum \alpha_j > \alpha_{ij}$.
In other words, the sparsity of the Dirichlet distribution associated with each non-root internal node is limited by the Dirichlet parameter corresponding to its parent.
\end{defn}

\begin{claim}
For any HPAM2 model based on a tree in which the Dirichlet parameters satisfy the limited-sparsity property, we have the following:
\begin{itemize}
  \item[(a)] If $i$ is a leaf node, then the maximum value of $C_{i,j}$ for $j \neq i$ is attained when $j$ corresponds to either the parent of $i$, a sibling of $i$, or a descendant of a sibling of $i$.
  \item[(b)] If $i$ is an internal node, then the maximum value of $C_{i,j}$ for $j \neq i$ is attained when $j$ corresponds to any descendant of $i$.
\end{itemize}
\label{claim:CMatrixProperty}
\end{claim}

We prove this claim using the following lemmas.

\begin{lemma}
Suppose we have three distinct nodes $a$, $b$, and $c$, where $\text{LCA}(a,b) = \text{LCA}(a,c)$.
Then $C_{a,b} = C_{a,c}$.
\label{lemma:equalLCA}
\end{lemma}

\begin{proof}
This lemma follows from the fact that, as long as $i \neq j$, the expression for $C_{i,j}$ derived in equation~\eqref{eqn:closedFormC} does not depend on any nodes except for those on the path from the root to $\text{LCA}(i,j)$.
\end{proof}

\begin{lemma}
Suppose the premises of Claim~\ref{claim:CMatrixProperty} hold, and suppose we have three distinct nodes $a$, $b$, and $c$, where $\text{LCA}(a,b)$ is a strict descendant of $\text{LCA}(a,c)$.
Then $C_{a,b} > C_{a,c}$.
\label{lemma:greaterLCA}
\end{lemma}

\begin{proof}
Let $\lambda = \text{LCA}(a,b)$ and let $\mu = \text{LCA}(a,c)$ such that $\lambda$ is a strict descendant of $\mu$.
Thus, the path from the root to $\lambda$ includes $\mu$.
Let the path from the root to $\lambda$ be denoted by
\begin{align}
\nu_0 \to \nu_1 \to \cdots \to \nu_{s_1} \to \cdots \to \nu_{s_2},
\end{align}
where $\nu_{s_1} = \mu$ and $\nu_{s_2} = \lambda$ for $s_1 < s_2$.

Using equation~\eqref{eqn:closedFormC} and the fact that $a \neq b$ and $a \neq c$, we can compute
\begin{align}
\frac{C_{a,b}}{C_{a,c}}
&=
\prod_{k=s_1+1}^{s_2}
\frac
  { \left( 1 + \alpha_{\nu_{k-1} \nu_k} \right) \sum \alpha_{\nu_k} }
  { \left( 1 + \sum \alpha_{\nu_k} \right) \alpha_{\nu_{k-1} \nu_k} }
\\
&=
\prod_{k=s_1+1}^{s_2}
\frac
  { \sum \alpha_{\nu_k} + \alpha_{\nu_{k-1} \nu_k} \sum \alpha_{\nu_k} }
  { \alpha_{\nu_{k-1} \nu_k} + \alpha_{\nu_{k-1} \nu_k} \sum \alpha_{\nu_k} }
\end{align}
%
Since we assumed that our model satisfies the limited-sparsity property, we know that $\sum \alpha_{\nu_k} > \alpha_{\nu_{k-1} \nu_k}$ for all $k$ such that $s_1 < k \leq s_2$; therefore, each factor in the above product is greater than $1$.
Furthermore, since $s_1 < s_2$, we know that the above product contains at least one factor.
Thus, we can conclude $C_{a,b} > C_{a,c}$.
\end{proof}

\begin{proof}[Proof of Claim~\ref{claim:CMatrixProperty}]
First, suppose $i$ is a leaf node, and let $S$ be the set containing the parent of $i$, the siblings of $i$, and the descendants of all siblings of $i$.
If $i$ is a leaf with no parent, then $i$ must be the only node in the tree, and Claim~\ref{claim:CMatrixProperty}(a) is vacuously true.
Thus, assume $i$ has a parent.
For any $j_1, j_2 \in S$, we know that $\text{LCA}(i, j_1) = \text{LCA}(i, j_2) = \text{parent}(i)$.
Therefore, we have $C_{i,j_1} = C_{i,j_2}$ by Lemma~\ref{lemma:equalLCA}.
Next, suppose $j$ is any node in $S$ and $k$ is any node outside of $S$ that is distinct from $i$.
Then, we know that $\text{LCA}(i,j) = \text{parent}(i)$, which must be a strict descendant of $\text{LCA}(i,k)$.
This is because $k \neq i$, $k \notin S$, and $i$ has no descendants of its own.
Therefore, $C_{i,j} > C_{i,k}$ using Lemma~\ref{lemma:greaterLCA}.
This proves Claim~\ref{claim:CMatrixProperty}(a).

Second, suppose $i$ is an internal node, and let $S$ be the set of all descendants of $i$ (excluding $i$ itself).
Since $i$ is an internal node, $S$ must be nonempty.
If $j_1, j_2 \in S$, then we know that $\text{LCA}(i, j_1) = \text{LCA}(i, j_2) = i$ and therefore $C_{i,j_1} = C_{i,j_2}$ by Lemma~\ref{lemma:equalLCA}.
Next, if $j$ is any node in $S$ and $k$ is any node outside of $S$ that is distinct from $i$, then we know $\text{LCA}(i,j) = i$ whereas $\text{LCA}(i,k)$ is a strict ancestor of $i$.
That is, $\text{LCA}(i,j)$ is a strict descendant of $\text{LCA}(i,k)$.
Then we can apply Lemma~\ref{lemma:greaterLCA} to obtain $C_{i,j} > C_{i,k}$.
This proves Claim~\ref{claim:CMatrixProperty}(b).
\end{proof}

%TODO: Derive an algorithm that can provably extract the correct tree structure given the matrix $C$.
%(Try using Aho's algorithm for tree construction based on triplet constraints?)

\end{document}
