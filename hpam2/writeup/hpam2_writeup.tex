\documentclass{article}

\usepackage{fullpage}
\usepackage[utf8]{inputenc} % allow utf-8 input
\usepackage[T1]{fontenc}    % use 8-bit T1 fonts
\usepackage{hyperref}       % hyperlinks
\usepackage{url}            % simple URL typesetting
\usepackage{booktabs}       % professional-quality tables
\usepackage{amsmath}
\usepackage{amssymb}
\usepackage{amsfonts}       % blackboard math symbols
\usepackage{mathtools}
\usepackage{nicefrac}       % compact symbols for 1/2, etc.
\usepackage{microtype}      % microtypography
\usepackage{algorithm}
\usepackage{algpseudocode}
\usepackage{graphicx}
\usepackage{float}
\usepackage{caption}
\usepackage{outlines}
\usepackage{tikz}
\usepackage{dsfont}
\usepackage{empheq}
\usepackage{xfrac}
\usepackage{enumitem}
\usepackage{amsthm}
%\usepackage{geometry}
%\geometry{margin=1.5in}

\newcommand{\nth}{^{\text{th}}}
\newcommand{\len}{\text{len}}
\newcommand{\indicator}{\mathds{1}}
\newcommand{\hackystatei}[1]{\State \parbox[t]{\dimexpr\linewidth-\algorithmicindent}{#1\strut}}
\newcommand{\hackystateii}[1]{\State \parbox[t]{\dimexpr\linewidth-\algorithmicindent-\algorithmicindent}{#1\strut}}
\newcommand{\argmax}{\mathop{\mathrm{argmax}}}
\newcommand{\argmin}{\mathop{\mathrm{argmin}}}

\newcommand{\Dirichlet}{\text{Dirichlet}}
\newcommand{\Categorical}{\text{Categorical}}
\newcommand{\Exit}{\textsc{Exit}}

%\allowdisplaybreaks

\title{Analysis of HPAM2 Model}

\author{
  Andrew Leverentz \\
  \texttt{aleveren@eng.ucsd.edu} \\
}

\date{}

\begin{document}

\maketitle

%%%%%%%%%%%%%%%%%%%%%%%%%%%

\section{Defining the HPAM2 Model}

In the hierarchical topic model HPAM2 (Hierarchical Pachinko Allocation Modeling, variant 2), we are given a rooted, directed acyclic graph $G$ and a collection of Dirichlet parameters.
Specifically, we have a vector $\alpha_\nu$ for each non-terminal node $\nu$, where the dimension of $\alpha_\nu$ is $1 + |\text{children}(\alpha_\nu)|$.
Note that $\alpha_\nu$ is not defined for terminal nodes (i.e., nodes with no outgoing edges).
We consider $\alpha_\nu$ to be indexed by the set $\{0\} \cup \text{children}(\alpha_\nu)$.
We also have a vector $\eta$ which is the parameter for a $V$-dimensional Dirichlet, where $V$ is the size of our vocabulary.

The generative model for sampling a corpus of documents is as follows:
For each node $\nu \in G$ (including terminal nodes), sample $\beta_\nu$ according to $\text{Dirichlet}(\eta)$.
For each document $d$ and for each non-terminal node $\nu$, sample $\theta_{d,\nu}$ according to $\text{Dirichlet}(\alpha_\nu)$.
Next, for each word-slot $n$ in document $d$ and for each non-terminal node $\nu$, sample $z_{d,\nu,n}$ according to $\text{Categorical}(\theta_{d,\nu})$; here, we are choosing whether to exit the DAG early (by selecting $0$), or to continue further down the DAG by selecting a child of $\nu$.
Next, we can define, for each document $d$ and each word-slot $n$, a path $\pi_{d,n}$ through $G$ beginning at the root node.
Specifically, the first element of $\pi_{d,n}$ is always the root node of $G$, and we repeatedly extend $\pi_{d,n}$ according to the sampled value of $z_{d,\nu,n}$, where $\nu$ represents the previously-selected node.
If we ever encounter a selection $z_{d,\nu,n} = 0$, we terminate the path early.
Finally, we sample a word from the vocabulary $t_{d,n}$ according to $\text{Dirichlet}(\beta_{\Exit(\pi_{d,n})})$.
Here, $\Exit(\pi_{d,n})$ is the node in $G$ at which the path $\pi_{d,n}$ terminates, which may or may not be a terminal node.

\section{Analysis of Topic Co-occurrence Probabilities in HPAM2}

Next, we compute topic co-occurrence probabilities for the HPAM2 model.
Specifically, we wish to know the probability that a single document contains some pair of topics $\nu_1$ and $\nu_2$:
\begin{align}
p(\Exit(\pi_{d,1}) = \nu_1,
  \Exit(\pi_{d,2}) = \nu_2)
&=
\sum_{
  \substack{
    \tilde\pi_1 : \Exit(\tilde\pi_1) = \nu_1
    \\
    \tilde\pi_2 : \Exit(\tilde\pi_2) = \nu_2
  }
}
p(\pi_{d,1} = \tilde\pi_1, \pi_{d,2} = \tilde\pi_2)
\end{align}
Next, let $\nu \to \nu' \in \tilde\pi_1$ denote the set of transitions $(\nu, \nu')$ appearing in $\tilde\pi_1$.
Furthermore, let $\nu \to * \in \tilde\pi_1 \cup \tilde\pi_2$ denote the set of nodes $\nu$ for which there exists some transition $\nu \to \nu'$ appearing in either $\tilde\pi_1$ or $\tilde\pi_2$.
Let $\int_{\tilde\theta \in \Theta(\tilde\pi_1, \tilde\pi_2)}$ denote an iterated integral over $\tilde\theta_\nu$ for each $\nu \to * \in \tilde\pi_1 \cup \tilde\pi_2$.
With this notation, we can write the co-occurrence probability of two paths as follows:
\begin{align}
p(\pi_{d,1} = \tilde\pi_1, \pi_{d,2} = \tilde\pi_2)
&=
\int_{\tilde\theta \in \Theta(\tilde\pi_1, \tilde\pi_2)}
  \left(
    \prod_{\nu \to * \in \tilde\pi_1 \cup \tilde\pi_2}
    p(\theta_{d,\nu} = \tilde\theta_\nu)
  \right)
  \\ &\phantom{=}\qquad \cdot
  \left(
    \prod_{\nu \to \nu' \in \tilde\pi_1}
    p(z_{d,\nu,1} = \nu' | \theta_{d,\nu} = \tilde\theta_\nu)
  \right)
  \\ &\phantom{=}\qquad \cdot
  \left(
    \prod_{\nu \to \nu' \in \tilde\pi_2}
    p(z_{d,\nu,2} = \nu' | \theta_{d,\nu} = \tilde\theta_\nu)
  \right)
\\
&=
\int_{\tilde\theta \in \Theta(\tilde\pi_1, \tilde\pi_2)}
  \left(
    \prod_{\nu \to * \in \tilde\pi_1 \cup \tilde\pi_2}
    \Dirichlet(\tilde\theta_\nu | \alpha_\nu)
  \right)
  %\\ &\phantom{=}\qquad \cdot
  \left(
    \prod_{\nu \to \nu' \in \tilde\pi_1}
    \tilde\theta_{\nu,\nu'}
  \right)
  %\\ &\phantom{=}\qquad \cdot
  \left(
    \prod_{\nu \to \nu' \in \tilde\pi_2}
    \tilde\theta_{\nu,\nu'}
  \right)
\end{align}
Then, if we define
\begin{align}
S(\nu) &= \{0\} \cup \text{children}(\nu), \\
N(\nu,\nu',\tilde\pi_1,\tilde\pi_2) &= \text{number of times $\nu \to \nu'$ appears in either $\tilde\pi_1$ or $\tilde\pi_2$}, \\
M(\nu,\tilde\pi_1,\tilde\pi_2) &= \text{number of times $\nu \to *$ appears in either $\tilde\pi_1$ or $\tilde\pi_2$},
\end{align}
we have
\begin{align}
p(\pi_{d,1} = \tilde\pi_1, \pi_{d,2} = \tilde\pi_2)
&=
\prod_{\nu \to * \in \tilde\pi_1 \cup \tilde\pi_2}
\left[
\frac
  {
    \prod_{\nu' \in S(\nu)}
    \prod_{k=0}^{N(\nu,\nu',\tilde\pi_1,\tilde\pi_2) - 1}
    (k + \alpha_{\nu,\nu'})
  }
  {
    \prod_{k=0}^{M(\nu,\tilde\pi_1,\tilde\pi_2) - 1}
    \left(
      k + \sum_{\nu' \in S(\nu)}\alpha_{\nu,\nu'}
    \right)
  }
\right]
\end{align}

Note that this formula easily generalizes beyond pairwise co-occurrences.
For example, we can compute three-way path co-occurrence probabilities as follows:
\begin{align}
p(\pi_{d,1} = \tilde\pi_1, \pi_{d,2} = \tilde\pi_2, \pi_{d,3} = \tilde\pi_3)
&=
\prod_{\nu \to * \in \tilde\pi_1 \cup \tilde\pi_2 \cup \tilde\pi_3}
\left[
\frac
  {
    \prod_{\nu' \in S(\nu)}
    \prod_{k=0}^{N(\nu,\nu',\tilde\pi_1,\tilde\pi_2,\tilde\pi_3) - 1}
    (k + \alpha_{\nu,\nu'})
  }
  {
    \prod_{k=0}^{M(\nu,\tilde\pi_1,\tilde\pi_2,\tilde\pi_3) - 1}
    \left(
      k + \sum_{\nu' \in S(\nu)}\alpha_{\nu,\nu'}
    \right)
  }
\right]
\end{align}
Furthermore, we can view the formula for single-path probabilities as a special case of the $n$-way co-occurrence formula:
\begin{align}
p(\pi_{d,1} = \tilde\pi_1)
&=
\prod_{\nu \to \nu' \in \tilde\pi_1}
\left[
\frac
  {
    \alpha_{\nu,\nu'}
  }
  {
    \sum_{\nu'' \in S(\nu)}\alpha_{\nu,\nu''}
  }
\right]
=
\prod_{\nu \to \nu' \in \tilde\pi_1}
E[\theta_{d,\nu}]_{\nu'}
\end{align}

\end{document}
