\usepackage{mathtools}
%\usepackage{algpseudocode}
\usepackage{listings}
\usepackage{tikz}
\usepackage[nomessages]{fp}
\usepackage{adjustbox}
\usepackage{outlines}
%\usepackage{times}
\usepackage{graphicx,color}
\usepackage{array,float}
\usepackage{url}
%\usepackage[usenames,dvipsnames]{xcolor}
\usepackage{amstext,amssymb,amsmath}
\usepackage{hyphenat}
\usepackage{amsthm}
\usepackage{verbatim}
\usepackage{bm}
\usepackage{paralist}
\usepackage{ulem}\normalem
\usepackage[numbers]{natbib}
\usepackage{todonotes}
\usepackage{paralist}
\usepackage{wrapfig}
\usepackage{hyperref}
\usepackage[noend]{algorithmic}
\usepackage{algorithm}
\usepackage{commath}
\usepackage{needspace}
\usepackage{mathrsfs}

\newcommand{\rb}[1]{\textcolor{blue}{RB:#1}}
\newcommand{\al}[1]{\textcolor{blue}{AL:#1}}
\newcommand{\cz}[1]{\textcolor{blue}{CZ:#1}}


\newtheorem{lem}{Lemma}[section]
\newtheorem{thm}[lem]{Theorem}
\newtheorem{cor}[lem]{Corollary}
\newtheorem{problem}[lem]{Problem}
\newtheorem{defn}[lem]{Definition}
\newtheorem{fact}[lem]{Fact}
\newtheorem{remark}{Remark}
\newtheorem{example}{Example}
\newtheorem{assumption}[lem]{Assumption}
\newtheorem{claim}[lem]{Claim}
\newcommand{\ip}[2]{\left\langle #1,#2 \right\rangle}

\usepackage{dsfont}

\newcommand{\err}{\mathrm{err}}
\newcommand{\emperr}{\widehat{\err}}
\newcommand{\pr}[2]{\underset{#1}{\mathbb{P}}\left[ #2 \right]}
\newcommand{\ex}[2]{\underset{#1}{\mathbb{E}}\left[ #2 \right]}

\newcommand{\R}{\mathbb{R}}
\newcommand{\N}{\mathbb{N}}
\newcommand{\B}{\mathrm{B}}

\newcommand{\deriv}[2]{\frac{d #1}{d #2}}
\newcommand{\pDeriv}[2]{\frac{\partial #1}{\partial #2}}
\newcommand{\pDerivTwo}[3]{\frac{\partial^2 #1}{\partial #2 \partial #3}}

\newcommand{\cond}{\,|\,}

\newcommand{\quickFig}[2]{ %
  \begin{figure}[H] %
  \centering %
  \includegraphics[width=#1 \linewidth,natwidth=600]{#2} %
  \end{figure} %
}

\newcommand{\nth}{^{\mathrm{th}}}

\DeclareMathOperator*{\argmax}{arg\,max}
\DeclareMathOperator*{\argmin}{arg\,min}
\DeclareMathOperator*{\argsup}{arg\,sup}
\DeclareMathOperator*{\arginf}{arg\,inf}

\newcommand{\domain}{\mathop{\mathrm{domain}}}

\newcommand{\VC}{\mathop{\mathrm{VC}}}
\newcommand{\var}{\mathop{\mathrm{var}}}
\newcommand{\cov}{\mathop{\mathrm{cov}}}
\newcommand{\sign}{\mathop{\mathrm{sign}}}
\newcommand{\eps}{\varepsilon}


\newcommand{\bx}{\mathbf{x}}
\newcommand{\by}{\mathbf{y}}
\newcommand{\bz}{\mathbf{z}}
\newcommand{\bw}{\mathbf{w}}
\newcommand{\bu}{\mathbf{u}}
\newcommand{\bv}{\mathbf{v}}
\newcommand{\bp}{\mathbf{p}}
\newcommand{\cC}{\mathcal{C}}
\newcommand{\cH}{\mathcal{H}}
\newcommand{\cG}{\mathcal{G}}
\newcommand{\cF}{\mathcal{F}}
\newcommand{\cA}{\mathcal{A}}
\newcommand{\cX}{\mathcal{X}}
\newcommand{\zeroB}{\ensuremath{\mathbf{0}}}

\newcommand{\bigO}[1]{O\left(#1\right)}
\newcommand{\indi}[1]{\mathbf{1}\left(#1\right)}

\newcommand{\midvert}{\mathrel{}\middle\vert\mathrel{}}

\newcommand{\expect}[1]{\mathbb{E} \left[ #1 \right]}
\newcommand{\expectCond}[2]{\mathbb{E}%
    \left[ #1 \midvert #2 \right]}
\newcommand{\expectOver}[2]{\mathbb{E}_{\substack{#1}}\left[ #2 \right]}
\newcommand{\expectOverCond}[3]{\mathbb{E}_{\substack{#1}}%
    \left[ #2 \midvert #3 \right]}

\newcommand{\prob}[1]{\mathbb{P} \left( #1 \right)}
\newcommand{\probCond}[2]{\mathbb{P}%
    \left( #1 \midvert #2 \right)}
\newcommand{\probOver}[2]{\mathbb{P}_{\substack{#1}}\left( #2 \right)}
\newcommand{\probOverCond}[3]{\mathbb{P}_{\substack{#1}}%
    \left( #2 \midvert #3 \right)}

\lstset{basicstyle=\ttfamily}

\usepackage{etoolbox}

\newcommand{\surround}[2][r]%
  {\ifstrequal{#1}{round}%
    {\left( #2 \right)}%
    {\ifstrequal{#1}{square}%
      {\left[ #2 \right]}%
      {\ifstrequal{#1}{curly}%
        {\left\{ #2 \right\}}%
        {\ifstrequal{#1}{angle}%
          {\left\langle #2 \right\rangle}%
          {\ifstrequal{#1}{|}%
            {\left\lvert #2 \right\rvert}%
            {\ifstrequal{#1}{||}%
              {\left\lVert #2 \right\rVert}%
              {\ifstrequal{#1}{floor}%
                {\left\lfloor #2 \right\rfloor}%
                {\ifstrequal{#1}{ceil}%
                  {\left\lceil #2 \right\rceil}%
                  {\ifstrequal{#1}{.}%
                    {\left. #2 \right.}%
                    {\left( #2 \right)}%
                  }%
                }%
              }%
            }%
          }%
        }%
      }%
    }%
  }

\newcommand{\innerProduct}[2]{\surround[angle]{{#1}, {#2}}}
